\chapter{Full-Surface Sliding Mode Controller (Four-State Surface)}
\label{ch:smc4}

\section{Base tracking errors}
The base terms in SMC4 are written in tracking-error form so the same controller covers both hold-at-center ($\theta_{\mathrm{ref}}=\mathrm{const}$) and commanded moves (nudge mode, time-varying $\theta_{\mathrm{ref}}$). Define
\[
\theta_{\mathrm{err}} \equiv \mathrm{diff}(\theta,\theta_{\mathrm{ref}}),\qquad
\dot{\theta}_{\mathrm{err}} \equiv \dot{\theta}-\dot{\theta}_{\mathrm{ref}},
\]
with the wrap-safe operator $\mathrm{diff}(\cdot,\cdot)$ from Chapter~\ref{ch:notation}. The reference acceleration $\ddot{\theta}_{\mathrm{ref}}$ appears as a feedforward term in the surface dynamics.

\section{Full-state sliding surface}
The full-surface controller augments the pendulum sliding surface with a base-centering term:
\begin{equation}
s \equiv \dot{\alpha} + \lambda_{\alpha}\alpha
  + k\Bigl(\dot{\theta}_{\mathrm{err}} + \lambda_{\theta}\theta_{\mathrm{err}}\Bigr),
\label{eq:smc4_surface}
\end{equation}
with design parameters $\lambda_{\alpha}>0$, $\lambda_{\theta}>0$ and coupling weight $k\ge 0$. For $k=0$, \eqref{eq:smc4_surface} reduces to the pendulum-only surface used in Chapter~\ref{ch:smc}.

\section{Surface dynamics and control effectiveness}
Differentiating \eqref{eq:smc4_surface} gives
\begin{equation}
\dot{s} = \ddot{\alpha} + \lambda_{\alpha}\dot{\alpha}
        + k\ddot{\theta}_{\mathrm{err}} + k\lambda_{\theta}\dot{\theta}_{\mathrm{err}}.
\label{eq:smc4_sdot_start}
\end{equation}
Using $\ddot{\theta}_{\mathrm{err}}=\ddot{\theta}-\ddot{\theta}_{\mathrm{ref}}$ and substituting the nonlinear relation \eqref{eq:smc_nonlinear_relation} yields
\begin{equation}
\dot{s}
=
\underbrace{\Bigl(A\sin\alpha+\sin\alpha\cos\alpha\,\dot{\theta}^2\Bigr)}_{\triangleq f_0(\alpha,\dot{\theta})}
\;+\;
\lambda_{\alpha}\dot{\alpha}
\;+\;
k\lambda_{\theta}\dot{\theta}_{\mathrm{err}}
\;-\;
k\ddot{\theta}_{\mathrm{ref}}
\;+\;
(k-B\cos\alpha)\ddot{\theta}.
\label{eq:smc4_sdot}
\end{equation}
Define the full drift term
\[
f_{\mathrm{full}} \equiv f_0(\alpha,\dot{\theta})
 + \lambda_{\alpha}\dot{\alpha}
 + k\lambda_{\theta}\dot{\theta}_{\mathrm{err}}
 - k\ddot{\theta}_{\mathrm{ref}},
\]
so that \eqref{eq:smc4_sdot} becomes
\[
\dot{s} = f_{\mathrm{full}} + (k-B\cos\alpha)\ddot{\theta}.
\]
The control effectiveness depends on the factor $(B\cos\alpha-k)$: if $k$ approaches $B\cos\alpha$, the controller becomes ill-conditioned (large accelerations are required to achieve a given $\dot{s}$). Therefore, the implementation enforces $k$ limits and a denominator margin (Section~\ref{sec:smc4_safeguards}).
Because upright-only operation enforces $|\alpha|<\SI{25}{\degree}$, $\cos\alpha \ge \cos(25^\circ)\approx 0.91$. With $B\approx 1.95$ and the nominal choice $k=0.50$, the nominal denominator satisfies $B\cos\alpha-k \gtrsim 1.27$, i.e., comfortably away from the singularity even before applying the additional margin enforcement.

\section{Reaching law and complete control law}
As in Chapter~\ref{ch:smc}, use the boundary-layer reaching law
\[
\dot{s} = -K\,\mathrm{sat}\!\left(\frac{s}{\phi}\right),\qquad K>0,\ \phi>0.
\]
Substituting into $\dot{s} = f_{\mathrm{full}} + (k-B\cos\alpha)\ddot{\theta}$ gives
\[
(B\cos\alpha-k)\ddot{\theta}
=
f_{\mathrm{full}} + K\,\mathrm{sat}\!\left(\frac{s}{\phi}\right),
\]
and therefore the full-surface sliding mode control law is
\begin{equation}
\boxed{
\ddot{\theta}
  =
  \frac{
    f_0(\alpha,\dot{\theta})
    + \lambda_{\alpha}\dot{\alpha}
    + k\lambda_{\theta}\dot{\theta}_{\mathrm{err}}
    - k\ddot{\theta}_{\mathrm{ref}}
    + K\,\mathrm{sat}\!\left(\dfrac{s}{\phi}\right)
  }{
    B\cos\alpha - k
  }
}
\label{eq:smc4_control_law}
\end{equation}
with $s$ given by \eqref{eq:smc4_surface}. Setting $k=0$ reduces \eqref{eq:smc4_control_law} to the pendulum-only SMC law \eqref{eq:smc_control_law}.

\section{Lyapunov stability analysis (reaching)}
As in the hybrid controller, the reaching law is designed to drive the sliding variable $s$ in \eqref{eq:smc4_surface} toward zero. Choose the Lyapunov candidate
\[
V(s)=\frac{1}{2}s^2,\qquad V\ge 0.
\]
With the boundary-layer reaching law $\dot{s}=-K\,\mathrm{sat}(s/\phi)$ \cite{utkin1992,slotine1991,Young1999},
\[
\dot{V}=s\dot{s}=-K\,s\,\mathrm{sat}\!\left(\frac{s}{\phi}\right)\le 0.
\]
Therefore $V$ never increases and $s$ is driven toward the sliding surface.
\begin{itemize}
  \item If $|s|>\phi$, then $\mathrm{sat}(s/\phi)=\mathrm{sgn}(s)$ and $\dot{V}=-K|s|<0$ (strong pull toward the boundary layer).
  \item If $|s|\le\phi$, then $\mathrm{sat}(s/\phi)=s/\phi$ and $\dot{V}=-(K/\phi)s^2\le 0$ (smooth convergence with reduced chattering near $s=0$).
\end{itemize}
When $s\approx 0$, the surface \eqref{eq:smc4_surface} enforces a coupled first-order relation between pendulum motion and the base tracking errors. The tuning parameters $(\lambda_{\alpha},\lambda_{\theta},k)$ set how strongly these errors are driven back toward zero in the sliding phase.

\section{Implementation safeguards and stepper-friendly shaping}
\label{sec:smc4_safeguards}
Compared to the hybrid controller, SMC4 is more sensitive to coupling because the term $k$ appears both in the surface and in the denominator $(B\cos\alpha-k)$. The implementation therefore applies a small set of safeguards to keep the denominator away from zero and to keep the stepper actuation within safe limits:
\begin{itemize}
  \item \textbf{Upright-only and cosine floor}: enforce $|\alpha|<\SI{25}{\degree}$ (upright-only operation) and use a protected $\cos_{\mathrm{safe}}=\mathrm{sgn}(\cos\alpha)\max(|\cos\alpha|,\cos_{\min})$ with $\cos_{\min}=0.2$ (keeps the denominator well-behaved).
  \item \textbf{Limit coupling}: enforce $k\le k_{\max}$ with $k_{\max}=1.20$ (prevents overly aggressive coupling).
  \item \textbf{Denominator margin}: enforce a minimum separation from the singularity,
    $|B\cos_{\mathrm{safe}}-k|\ge \mathrm{den}_{\min}$ with $\mathrm{den}_{\min}=0.60$
    (prevents ill-conditioning and large acceleration demands).
  \item \textbf{Smooth engage}: ramp $k$ from 0 to its configured value over $T_k=\SI{200}{ms}$ after engage (avoids sudden transients).
  \item \textbf{Practical clamps and shaping}: clamp $|\theta_{\mathrm{err}}|\le\theta_{\mathrm{err},\max}=\SI{78}{\degree}$ and $|\dot{\theta}_{\mathrm{err}}|\le\dot{\theta}_{\mathrm{err},\max}=\SI{200}{\degree\per\second}$, clamp the internal $\dot{\alpha}$ used in the computation to $|\dot{\alpha}|\le\dot{\alpha}_{\mathrm{ctrl},\max}=\SI{200}{\degree\per\second}$, and apply tighter caps $u_{\max,\mathrm{SMC4}}=\SI{12000}{\step\per\second\squared}$ and $\dot{\theta}_{\max,\mathrm{SMC4}}=\SI{2000}{\step\per\second}$ (reduces spikes and stall/missed-step events).
\end{itemize}

\subsection{Gated base-centering assist (implemented)}
Although the surface \eqref{eq:smc4_surface} includes base error terms, a small gated base-centering assist (of the form \eqref{eq:smc_base_assist}) is also applied to prevent slow base drift once the pendulum is already near upright. The assist is scaled by $\gamma$ (default $\gamma=1.0$, range $0\ldots 2$) and enabled only after a grace period $T_g=\SI{200}{ms}$ and when $|\alpha|<\alpha_{\gamma}=\SI{5}{\degree}$ and $|\dot{\alpha}|<\dot{\alpha}_{\gamma,\mathrm{SMC4}}=\SI{250}{\degree\per\second}$. During large recovery transients it remains inactive.

\section{Parameters used in experiments}
For the dataset analyzed in Chapter~\ref{ch:results}, the full-surface controller used:
\begin{table}[htbp]
  \centering
  \small
  \setlength{\tabcolsep}{5pt}
  \caption{Full-surface sliding mode controller parameters used in experiments.}
  \label{tab:smc4_params}
  \begin{tabular}{lrl}
    \toprule
    Parameter & Value & Meaning \\
    \midrule
    $\lambda_{\alpha}$ & 15\ \si{s^{-1}} & pendulum surface slope \\
    $K$ & 800\ \si{\degree\per\second\squared} & reaching gain \\
    $\phi$ & 50\ \si{\degree\per\second} & boundary layer thickness \\
    $k$ & 0.50 & base coupling weight in \eqref{eq:smc4_surface} \\
    $\lambda_{\theta}$ & 2.0\ \si{s^{-1}} & base error shaping in \eqref{eq:smc4_surface} \\
    \midrule
    $k_{\max}$ & 1.20 & hard coupling limit \\
    $\mathrm{den}_{\min}$ & 0.60 & minimum denominator margin \\
    coupling ramp & \SI{200}{ms} & ramp-in duration after engage \\
    \bottomrule
  \end{tabular}
\end{table}
As with hybrid SMC, upright-only validity and extreme-rate protections are enabled (Chapter~\ref{ch:hardware}), and the final commanded acceleration is clamped before conversion to stepper units.

\ObservedBox{See Figures~\ref{fig:hold_smc4_timeseries}, \ref{fig:tap_smc4_timeseries} and Tables in Chapter~\ref{ch:results}.}

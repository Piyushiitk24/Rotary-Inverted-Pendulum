\chapter{Conclusion and Future Work}

\section{Conclusion}
This project developed a complete upright-only control stack for a custom stepper-driven rotary (Furuta) inverted pendulum:
\begin{itemize}
  \item A first-principles nonlinear model was derived and numerically specialized for the physical rig (Chapter~\ref{ch:modelling}).
  \item A practical real-time firmware pipeline was implemented, including robust derivative estimation, wrap-safe angle handling, glitch rejection, engage logic, soft limits, and stall diagnostics (Chapters~\ref{ch:measurement}--\ref{ch:actuation}).
  \item Three upright controllers were implemented and compared: linear full-state feedback, hybrid sliding mode control, and full-surface sliding mode control (Chapters~\ref{ch:lin}--\ref{ch:smc4}).
  \item Reference tracking while balancing (``nudge mode'') was demonstrated in the linear controller, converting the system from a regulator to a servo without sacrificing upright stability (Chapter~\ref{ch:lin} and Chapter~\ref{ch:results}).
\end{itemize}

Across the pinned dataset (Chapter~\ref{ch:experiments}), the linear and hybrid SMC controllers achieved stable hold-at-center behavior and provided a strong baseline for comparison. SMC4 demonstrated upright stabilization in undisturbed conditions, but its disturbance behavior highlights an important practical conclusion: on a stepper-actuated platform, aggressive nonlinear control effort can be limited by missed steps and stall phenomena even when the underlying control law is theoretically well-posed.

\section{Future work}
\begin{itemize}
  \item Increase the number of trials per mode for statistical defensibility.
  \item Add explicit disturbance markers in the logger (so tap timing can be aligned deterministically).
  \item Increase logging rate for higher-frequency effort/chatter characterization (beyond 25~Hz).
  \item Investigate actuator upgrades or torque-based actuation to reduce stall/tick events under aggressive SMC4 behavior.
\end{itemize}

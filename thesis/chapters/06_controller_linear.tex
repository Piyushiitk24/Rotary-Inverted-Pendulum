\chapter{Linear Full-State Feedback Controller}
\label{ch:lin}

\section{Linearized plant for control design}
The linear controller is designed about the upright equilibrium ($\alpha \approx 0$) using the acceleration-input formulation $u \equiv \ddot{\theta}$ (Chapter~\ref{ch:actuation}). Using the linearized model derived in Chapter~\ref{ch:modelling}, the reduced upright dynamics can be written in the state variables
\[
\mathbf{x} \;=\; \begin{bmatrix}\theta & \alpha & \dot{\theta} & \dot{\alpha}\end{bmatrix}^{\mathsf{T}}
\;\equiv\; \begin{bmatrix}x_1 & x_2 & x_3 & x_4\end{bmatrix}^{\mathsf{T}}
\]
as
\begin{align}
\dot{x}_1 &= x_3, &
\dot{x}_2 &= x_4, &
\dot{x}_3 &= u, &
\dot{x}_4 &= A\,x_2 - B\,u.
\label{eq:lin_plant}
\end{align}
Here $A$ and $B$ are the identified constants (Chapter~\ref{ch:modelling}), with numerical values for this rig
\[
A = 100.8,\qquad B = 1.952.
\]
In state-space form,
\[
\dot{\mathbf{x}} = \mathbf{A}\mathbf{x} + \mathbf{B}u,\qquad
\mathbf{A}=
\begin{bmatrix}
0&0&1&0\\
0&0&0&1\\
0&0&0&0\\
0&A&0&0
\end{bmatrix},\qquad
\mathbf{B}=
\begin{bmatrix}
0\\0\\1\\-B
\end{bmatrix}.
\]
Linear state-feedback designs (often via LQR) are a common baseline for rotary inverted pendulum stabilization \cite{Park2011,Chawla2018}.

\section{Full-state feedback by pole placement (coefficient matching)}
Choose a state feedback law
\begin{equation}
u = -\mathbf{K}\mathbf{x},\qquad
\mathbf{K}=\begin{bmatrix}k_\theta & k_\alpha & k_{\dot{\theta}} & k_{\dot{\alpha}}\end{bmatrix}.
\label{eq:lin_state_feedback}
\end{equation}
Substituting \eqref{eq:lin_state_feedback} into \eqref{eq:lin_plant} yields the closed-loop matrix
\[
\mathbf{A}_{\mathrm{cl}}=\mathbf{A}-\mathbf{B}\mathbf{K}
=
\begin{bmatrix}
0 & 0 & 1 & 0\\
0 & 0 & 0 & 1\\
-k_\theta & -k_\alpha & -k_{\dot{\theta}} & -k_{\dot{\alpha}}\\
B k_\theta & A + B k_\alpha & B k_{\dot{\theta}} & B k_{\dot{\alpha}}
\end{bmatrix}.
\]
Eliminating $\theta$ from the scalar equations gives a fourth-order characteristic polynomial of the form
\begin{equation}
s^4 + a_3 s^3 + a_2 s^2 + a_1 s + a_0 = 0,
\label{eq:lin_charpoly}
\end{equation}
with coefficients expressed directly in terms of the gains:
\begin{align}
a_3 &= k_{\dot{\theta}} - B k_{\dot{\alpha}}, &
a_2 &= k_{\theta} - A - B k_{\alpha}, &
a_1 &= -A k_{\dot{\theta}}, &
a_0 &= -A k_{\theta}.
\label{eq:lin_coeffs}
\end{align}
This form enables hand-calculable coefficient matching \cite{ogata_modern_control}: pick a desired polynomial and solve \eqref{eq:lin_coeffs} for the four gains.

\subsection{Desired poles via two second-order factors}
The desired closed-loop polynomial is chosen as a product of two standard second-order factors:
\begin{equation}
\bigl(s^2+2\zeta_1\omega_1 s+\omega_1^2\bigr)\,
\bigl(s^2+2\zeta_2\omega_2 s+\omega_2^2\bigr)
= s^4 + \tilde a_3 s^3 + \tilde a_2 s^2 + \tilde a_1 s + \tilde a_0,
\label{eq:lin_desired_poly}
\end{equation}
where $(\omega_1,\zeta_1)$ correspond to a fast ``pendulum mode'' and $(\omega_2,\zeta_2)$ to a slower ``base-centering mode''. Expanding gives
\begin{align}
\tilde a_3 &= 2(\zeta_1\omega_1+\zeta_2\omega_2),\nonumber\\
\tilde a_2 &= \omega_1^2+\omega_2^2+4\zeta_1\zeta_2\omega_1\omega_2,\nonumber\\
\tilde a_1 &= 2(\zeta_1\omega_1\omega_2^2+\zeta_2\omega_2\omega_1^2),\nonumber\\
\tilde a_0 &= \omega_1^2\omega_2^2.
\label{eq:lin_desired_coeffs}
\end{align}

\subsection{Closed-form gains}
Setting $(a_0,a_1,a_2,a_3)=(\tilde a_0,\tilde a_1,\tilde a_2,\tilde a_3)$ and solving \eqref{eq:lin_coeffs} yields, in sequence,
\begin{align}
k_{\theta} &= -\frac{\tilde a_0}{A}, &
k_{\dot{\theta}} &= -\frac{\tilde a_1}{A},\nonumber\\
k_{\dot{\alpha}} &= \frac{k_{\dot{\theta}}-\tilde a_3}{B}, &
k_{\alpha} &= \frac{k_{\theta}-A-\tilde a_2}{B}.
\label{eq:lin_gain_formulas}
\end{align}

\subsection{Numerical example and unit conversion}
Using $A=100.8$, $B=1.952$ and choosing a fast mode $(\omega_1,\zeta_1)=(15,0.8)$ and a slow mode $(\omega_2,\zeta_2)=(1,1)$ (rad/s), the desired coefficients are
\[
\tilde a_3=26,\qquad \tilde a_2=274,\qquad \tilde a_1=474,\qquad \tilde a_0=225,
\]
and the resulting gains from \eqref{eq:lin_gain_formulas} are
\[
\mathbf{K}=\begin{bmatrix}
-2.232 & -193.15 & -4.702 & -15.729
\end{bmatrix}.
\]
For implementation with degree states and stepper units, define the constant conversion $k_s$ (units \si{\step\per\degree}; Chapter~\ref{ch:notation}). The commanded acceleration in stepper units is
\[
u_{\mathrm{steps}} = k_s\,u_{\deg}.
\]
Therefore, the gain vector expressed in stepper acceleration units is
\[
\mathbf{K}_{\mathrm{steps}} = k_s\,\mathbf{K}.
\]
With $k_s=4.444$ (this rig), the example yields
\[
\mathbf{K}_{\mathrm{steps}}=
\begin{bmatrix}
-9.92 & -858.4 & -20.9 & -69.9
\end{bmatrix}.
\]
The sign of the gains depends on the chosen $\alpha$ and motor direction conventions; the implementation enforces consistency via a sign-validation procedure (Chapter~\ref{ch:notation}) and a single controller sign on the final acceleration command.

\section{Reference tracking and ``nudge'' servo behavior}
Rather than regulating the base to $\theta=0$ always, the implemented controller uses reference tracking:
\[
\theta_{\mathrm{err}} = \mathrm{diff}(\theta,\theta_{\mathrm{ref}}),\qquad
\dot{\theta}_{\mathrm{err}} = \dot{\theta} - \dot{\theta}_{\mathrm{ref}},
\]
with $\mathrm{diff}(\cdot,\cdot)$ the wrap-safe difference operator (Chapter~\ref{ch:notation}). The reference-tracking full-state feedback law is
\begin{equation}
u_{\mathrm{cmd}} =
u_{\mathrm{ff}} +
K_\theta\,\theta_{\mathrm{err}} + K_\alpha\,\alpha +
K_{\dot{\theta}}\,\dot{\theta}_{\mathrm{err}} + K_{\dot{\alpha}}\,\dot{\alpha},
\label{eq:lin_ref_tracking}
\end{equation}
with feedforward term $u_{\mathrm{ff}}=\ddot{\theta}_{\mathrm{ref}}$ provided by the reference generator.

\subsection*{Commanded base position while balancing (``nudge mode'')}
The reference $\theta_{\mathrm{ref}}$ is generated by a velocity- and acceleration-limited trapezoidal profile driven by a user-specified base angle target. This turns the upright controller into a practical \emph{servo} while maintaining balance.

Because aggressive base motion can destabilize the pendulum, the reference generator includes a pause/resume mechanism: if the pendulum leaves a stability window during a move, the profile is paused and the reference is frozen (without ``chasing'' the measured $\theta$). Once the pendulum returns to the stability window, the profile resumes.
\par
In implementation, the move stability gate is defined as $|\alpha|<\alpha_{\mathrm{move,win}}$ with $\alpha_{\mathrm{move,win}}=\SI{5}{\degree}$ and $|\dot{\alpha}|<\dot{\alpha}_{\mathrm{move,win}}$ with $\dot{\alpha}_{\mathrm{move,win}}=\SI{80}{\degree\per\second}$. The gate is enabled only after a post-engage grace $T_g=\SI{200}{ms}$. Stability/instability is debounced for $N_{\mathrm{stable}}=N_{\mathrm{unstable}}=10$ consecutive control ticks (\SI{50}{ms} at \SI{200}{Hz}) before pausing/resuming the profile. For safety, user targets are also bounded by $|\theta_{\mathrm{target}}|\le \theta_{\mathrm{target,max}}=\SI{35}{\degree}$.

\section{Implementation details (stepper-oriented)}
Several pragmatic additions are implemented to improve repeatability on real hardware:
\begin{itemize}
  \item \textbf{Deadband on $\alpha$}: $\alpha_{\mathrm{db}}=\SI{0.5}{\degree}$ reduces noise-chasing near upright.
  \item \textbf{Acceleration saturation and engage ramp}: the commanded acceleration is saturated to $u_{\max}=\SI{20000}{\step\per\second\squared}$ (and commanded speed to $\dot{\theta}_{\max}=\SI{15000}{\step\per\second}$), and multiplied by a linear engage ramp of duration $T_{\mathrm{ramp}}=\SI{100}{ms}$ after controller engagement.
  \item \textbf{Velocity leak (linear-only)}: $\lambda_{\mathrm{leak}}=\SI{2.0}{s^{-1}}$ is enabled only when $|\alpha|<\alpha_{\mathrm{leak,win}}=\SI{5}{\degree}$ and $|\theta_{\mathrm{err}}|<\theta_{\mathrm{leak,win}}=\SI{20}{\degree}$ to suppress slow drift (disabled in nonlinear modes).
  \item \textbf{Auto-trim of upright reference}: enabled only when $|\alpha|<\alpha_{\mathrm{trim,win}}=\SI{3}{\degree}$ and $|\dot{\alpha}|<\dot{\alpha}_{\mathrm{trim,win}}=\SI{50}{\degree\per\second}$; integrates at $\lambda_{\mathrm{trim}}=\SI{0.20}{s^{-1}}$ and is clamped to $|\alpha_{\mathrm{trim}}|\le \alpha_{\mathrm{trim,max}}=\SI{1.5}{\degree}$ relative to the calibration reference.
  \item \textbf{Soft limits}: outward velocity commands are suppressed within a margin $\Delta_\theta=\SI{2}{\degree}$ of the mechanical base limit.
\end{itemize}

\section{Parameters used in experiments}
For the dataset analyzed in Chapter~\ref{ch:results}, the linear controller used the following parameters (printed by the firmware at trial start):
\begin{table}[htbp]
  \centering
  \small
  \setlength{\tabcolsep}{5pt}
  \caption{Linear controller gains used in experiments (stepper acceleration units).}
  \label{tab:lin_gains}
  \begin{tabular}{lrrl}
    \toprule
    Parameter & Value & Units & Meaning \\
    \midrule
    $K_{\theta}$ & 31.5 & \si{\step\per\second\squared\per\degree} & base/arm position gain \\
    $K_{\alpha}$ & 725.6 & \si{\step\per\second\squared\per\degree} & pendulum angle gain \\
    $K_{\dot{\theta}}$ & 28.3 & \si{\step\per\second\per\degree} & base/arm rate gain \\
    $K_{\dot{\alpha}}$ & 73.5 & \si{\step\per\second\per\degree} & pendulum rate gain \\
    \bottomrule
  \end{tabular}
\end{table}
Derivative estimation uses the filtered differentiator described in Chapter~\ref{ch:measurement} with cutoff $\omega_c=\SI{450}{rad/s}$.

\ObservedBox{See Figures~\ref{fig:hold_lin_timeseries}, \ref{fig:nudge_lin_tracking}, \ref{fig:tap_lin_timeseries} and Tables in Chapter~\ref{ch:results}.}

\chapter{Firmware Pipeline: Actuation and Safety}
\label{ch:actuation}

\section{Stepper actuation pipeline}
The stepper driver is commanded in continuous-velocity mode. All controllers compute a base acceleration command, which is then mapped to the motor direction convention and integrated to produce the commanded step rate:
\[
u \equiv \ddot{\theta}\ \xrightarrow{\text{controller}}\ u_{\mathrm{cmd}}
\xrightarrow{\text{sign map}}\ u_{\mathrm{cmd,motor}}
\xrightarrow{\int}\ \dot{\theta}_{\mathrm{cmd}}.
\]
Here $u_{\mathrm{cmd}}$ is expressed as a step-acceleration command (units \si{\step\per\second\squared}), and $\dot{\theta}_{\mathrm{cmd}}$ is the commanded step rate (units \si{\step\per\second}).
The integration is implemented as:
\[
\dot{\theta}_{\mathrm{cmd}} \leftarrow \dot{\theta}_{\mathrm{cmd}} + \bigl(\ddot{\theta}_{\mathrm{cmd}} - \lambda_{\mathrm{leak}}\dot{\theta}_{\mathrm{cmd}}\bigr)\,T,
\]
where $\lambda_{\mathrm{leak}}$ is optionally enabled only in the \emph{linear} controller (Chapter~\ref{ch:lin}) to suppress slow drift.

\paragraph{Engage ramp.}
On transition into the active balancing state, the commanded acceleration is multiplied by an engage ramp to avoid impulsive start-up commands while derivatives and references are initializing. In firmware this ramp is a linear \SI{100}{ms} rise:
\[
u_{\mathrm{cmd}} \leftarrow u_{\mathrm{cmd}}\cdot \min\!\left(\frac{t-t_{\mathrm{engage}}}{\SI{100}{ms}},1\right).
\]

\paragraph{Stop-before-reverse.}
To avoid direction chatter and mechanical jerk near zero, the firmware enforces a stop-before-reverse policy: if the desired direction changes sign while the motor is running, a deceleration stop is commanded immediately and the reverse command can only occur on a subsequent control tick ($\ge\SI{5}{ms}$ later at \SI{200}{Hz}).

\section{Soft limits and failure modes}
The firmware includes several safety layers to make experiments repeatable on a stepper-driven rig:
\begin{itemize}
  \item \textbf{Soft base limits} near mechanical bounds (prevents ``pushing into the stop'').
  \item \textbf{Stall detection} when the motor is commanded but the base does not move (prints diagnostics for stall/missed-step events).
  \item \textbf{Predictable disarm/idle behavior} so runs can be restarted quickly without power-cycling.
\end{itemize}

\paragraph{Soft limits.}
In addition to the hard base angle limit $|\theta|>\SI{80}{\degree}$ (Chapter~\ref{ch:hardware}), the firmware applies a soft limit margin of \texttt{LIM\_MARGIN\_DEG}=\SI{2}{\degree}. If the measured base angle is within this margin (i.e., $|\theta|>\SI{78}{\degree}$) and the commanded velocity would push further outward, the velocity command is suppressed (clamped to zero). This reduces the probability of ``hitting the stop'' while the pendulum is still upright.

\paragraph{Stall detection (stall/missed-step events).}
In some runs (especially under aggressive disturbance recovery), the stepper can miss steps or fail to move despite a nonzero commanded velocity. The firmware detects this by monitoring the measured base angle change $\Delta\theta$ over each control tick while the motor is commanded above \texttt{BASE\_STALL\_CMD\_HZ}=\SI{500}{\step\per\second}. If motion is expected but the wrapped base angle change remains below \texttt{BASE\_STALL\_MIN\_DTHETA\_DEG}=\SI{0.10}{\degree} for \texttt{BASE\_STALL\_WARN\_TICKS}=30 consecutive control ticks (i.e., \SI{150}{ms} at \SI{200}{Hz}), a warning diagnostic is printed. The warning includes AS5600 base-sensor magnet diagnostics (status bits, AGC, magnitude) to help distinguish true motor/driver dropouts from sensor mounting issues.

\paragraph{Disarm and repeatability.}
When a hard limit is exceeded for \texttt{FALL\_COUNT\_REQ}=3 consecutive control ticks (debounced; \SI{15}{ms} at \SI{200}{Hz}), the firmware disables stepper outputs and returns to an idle state. The system remains ready to re-engage once the pendulum is returned upright, enabling rapid repeated trials without power-cycling.

\section{Auto-trim and velocity leak}
Two slow ``bias killers'' are implemented to improve repeatability under sensor offsets and small persistent disturbances:
\begin{itemize}
  \item \textbf{Auto-trim of the upright reference}: when stable and not saturated, the pendulum reference is updated at \texttt{ALPHA\_TRIM\_RATE}=\SI{0.20}{s^{-1}} and enabled only when $|\alpha|<\texttt{ALPHA\_TRIM\_ALPHA\_WINDOW\_DEG}=\SI{3}{\degree}$ and $|\dot{\alpha}|<\texttt{ALPHA\_TRIM\_ALPHADOT\_WINDOW\_DEG\_S}=\SI{50}{\degree\per\second}$. The total trim offset is clamped to $\pm\texttt{ALPHA\_TRIM\_MAX\_DEG}=\SI{1.5}{\degree}$ relative to the calibration reference.
  \item \textbf{Velocity leak (linear-only)}: the acceleration integrator uses a leaky term with \texttt{VEL\_LEAK}=\SI{2.0}{s^{-1}}, enabled only when near upright ($|\alpha|<\texttt{VEL\_LEAK\_ALPHA\_WINDOW\_DEG}=\SI{5}{\degree}$) and near the base reference ($|\theta_{\mathrm{err}}|<\texttt{VEL\_LEAK\_THETA\_WINDOW\_DEG}=\SI{20}{\degree}$). This reduces slow random-walk drift due to static bias in the acceleration command.
\end{itemize}
The velocity leak is \emph{disabled} in nonlinear modes. In SMC controllers, the acceleration command is derived directly from a nonlinear reaching law; adding a leak alters the effective input channel and can undermine the intended behavior.

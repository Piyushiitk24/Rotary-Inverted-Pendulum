\chapter{Experiments and Data Processing}
\label{ch:experiments}

\section{Dataset definition}
All results in this report are generated from a fixed dataset of experimental runs (summarized in Table~\ref{tab:dataset_manifest}). This avoids ambiguity and makes the analysis reproducible.

\subsection*{Report-facing run identifiers}
For readability, the internal run labels used by the analysis pipeline are mapped to report-facing identifiers:
\begin{itemize}
  \item Hold-at-center: \textbf{H1--H3}
  \item Commanded reference tracking (nudge): \textbf{N1}
  \item Finger tap disturbance: \textbf{T1--T3}
\end{itemize}
Each experimental run is archived using a timestamp-coded run identifier. Table~\ref{tab:dataset_manifest} lists the run identifiers used for this report.

\InputIfExists{tables/tab_dataset_manifest.tex}{\emph{(Run the export script to generate the dataset table.)}}

\section{Controller parameters used in experiments}
For completeness, Table~\ref{tab:controller_params_overview} summarizes the numeric controller and pipeline parameters used to generate the results in Chapter~\ref{ch:results}. These values correspond to the fixed configurations used during the pinned dataset runs (H1--H3, N1, T1--T3).
\begingroup
\small
\setlength{\tabcolsep}{4pt}
\renewcommand{\arraystretch}{0.95}
\begin{longtable}{@{}p{0.16\linewidth}p{0.26\linewidth}p{0.12\linewidth}p{0.16\linewidth}p{0.24\linewidth}@{}}
  \caption{Controller and pipeline parameters used in the pinned experimental dataset.}
  \label{tab:controller_params_overview}\\
  \toprule
  Category & Parameter & Value & Units & Notes \\
  \midrule
  \endfirsthead
  \toprule
  Category & Parameter & Value & Units & Notes \\
  \midrule
  \endhead
  \midrule
  \multicolumn{5}{r}{\emph{(Continued on next page)}}\\
  \midrule
  \endfoot
  \bottomrule
  \endlastfoot

  Shared & Stepper conversion $k_s$ & 4.444 & \si{\step\per\degree} & used to convert degrees to step units \\
  Shared & Control tick & 200 & Hz & fixed-period loop \\
  Shared & CSV logging rate & 50 & Hz & decimated from control tick \\
  Shared & Nominal tick period & 5000 & \si{\micro s} & \texttt{LOOP\_US} (target loop period) \\
  Shared & Tick overrun threshold & 6000 & \si{\micro s} & counted if $dt>\texttt{LOOP\_US}+1000$ \\
  Shared & Base angle limit & 80 & \si{\degree} & hard disarm beyond this bound \\
  Shared & Pendulum angle limit & 30 & \si{\degree} & hard disarm beyond this bound \\
  Shared & Hard limit debounce & 3 & ticks & \texttt{FALL\_COUNT\_REQ} (= \SI{15}{ms} at \SI{200}{Hz}) \\
  Shared & Soft limit margin & 2 & \si{\degree} & \texttt{LIM\_MARGIN\_DEG} (clamps outward motion for $|\theta|>\SI{78}{\degree}$) \\
  Shared & Engage ramp duration & 100 & ms & linear ramp applied after \texttt{ENGAGED} \\
  Shared & Global acceleration cap & 20000 & \si{\step\per\second\squared} & \texttt{MAX\_ACC\_STEPS} \\
  Shared & Global speed cap & 15000 & \si{\step\per\second} & \texttt{MAX\_SPEED\_HZ} \\
  Shared & Stall detect min cmd speed & 500 & \si{\step\per\second} & \texttt{BASE\_STALL\_CMD\_HZ} (expect-motion threshold) \\
  Shared & Stall detect min motion & 0.10 & \si{\degree} & \texttt{BASE\_STALL\_MIN\_DTHETA\_DEG} per control tick \\
  Shared & Stall detect persistence & 30 & ticks & \texttt{BASE\_STALL\_WARN\_TICKS} (= \SI{150}{ms}) \\
  Shared (nonlinear) & Upright-only validity $|\alpha|$ & 25 & \si{\degree} & abort to enforce upright-only operation \\
  Shared (nonlinear) & Extreme-rate abort $|\dot{\alpha}|$ & 250 & \si{\degree\per\second} & debounced, prevents rail commands \\
  Shared (nonlinear) & Extreme-rate abort debounce & 3 & ticks & \texttt{SMC\_ALPHADOT\_ABORT\_TICKS\_REQ} \\
  Shared (nonlinear) & Extreme-rate abort grace & 200 & ms & \texttt{SMC\_ABORT\_GRACE\_MS} post-engage \\
  Shared (nonlinear) & Extreme-rate abort min $|\alpha|$ & 5 & \si{\degree} & \texttt{SMC\_ALPHADOT\_ABORT\_MIN\_ALPHA\_DEG} \\
  Shared (nonlinear) & Cosine floor $\cos_{\min}$ & 0.2 & -- & \texttt{SMC\_COS\_MIN} in $\cos_{\mathrm{safe}}$ \\

  \addlinespace
	  Linear & $K_{\theta}$ & 31.5 & \si{\step\per\second\squared\per\degree} & base/arm position gain \\
	  Linear & $K_{\alpha}$ & 725.6 & \si{\step\per\second\squared\per\degree} & pendulum angle gain \\
	  Linear & $K_{\dot{\theta}}$ & 28.3 & \si{\step\per\second\squared}\,per\,(\si{\degree\per\second}) & base/arm rate gain \\
	  Linear & $K_{\dot{\alpha}}$ & 73.5 & \si{\step\per\second\squared}\,per\,(\si{\degree\per\second}) & pendulum rate gain \\
	  Linear & Derivative filter cutoff $\omega_c$ & 450 & \si{rad\per\second} & bilinear (Tustin) filtered differentiator \\
	  Linear & $\alpha$ deadband & 0.5 & \si{\degree} & \texttt{ALPHA\_DEADBAND\_DEG} \\
	  Linear & Velocity leak coefficient $\lambda_{\mathrm{leak}}$ & 2.0 & \si{s^{-1}} & enabled only near upright/center \\
	  Linear & Velocity leak window $|\alpha|$ & 5 & \si{\degree} & \texttt{VEL\_LEAK\_ALPHA\_WINDOW\_DEG} \\
	  Linear & Velocity leak window $|\theta_{\mathrm{err}}|$ & 20 & \si{\degree} & \texttt{VEL\_LEAK\_THETA\_WINDOW\_DEG} \\
	  Linear & Upright reference trim rate & 0.20 & \si{s^{-1}} & slow bias compensation (auto-trim) \\
	  Linear & Auto-trim clamp & 1.5 & \si{\degree} & \texttt{ALPHA\_TRIM\_MAX\_DEG} from calibration \\
	  Linear & Auto-trim window $|\alpha|$ & 3 & \si{\degree} & \texttt{ALPHA\_TRIM\_ALPHA\_WINDOW\_DEG} \\
	  Linear & Auto-trim window $|\dot{\alpha}|$ & 50 & \si{\degree\per\second} & \texttt{ALPHA\_TRIM\_ALPHADOT\_WINDOW\_DEG\_S} \\

  \addlinespace
  Hybrid SMC & Surface slope $\lambda$ & 15 & \si{s^{-1}} & $s=\dot{\alpha}+\lambda\alpha$ \\
  Hybrid SMC & Reaching gain $K$ & 800 & \si{\degree\per\second\squared} & boundary-layer reaching law \\
  Hybrid SMC & Boundary layer $\phi$ & 50 & \si{\degree\per\second} & reduces chattering \\
  Hybrid SMC & Base assist scale $\gamma$ & 1.0 & -- & gated base-centering assist \\
  Hybrid SMC & Base assist gate $|\alpha|$ & 5 & \si{\degree} & \texttt{SMC\_BASE\_GATE\_ALPHA\_WINDOW\_DEG} \\
  Hybrid SMC & Base assist gate $|\dot{\alpha}|$ & 150 & \si{\degree\per\second} & \texttt{SMC\_BASE\_GATE\_ALPHADOT\_WINDOW\_DEG\_S} \\

  \addlinespace
  Full-surface SMC & Pendulum slope $\lambda_{\alpha}$ & 15 & \si{s^{-1}} & part of four-state surface \\
  Full-surface SMC & Base slope $\lambda_{\theta}$ & 2.0 & \si{s^{-1}} & shapes base tracking term \\
  Full-surface SMC & Coupling weight $k$ & 0.50 & -- & couples base term into surface \\
  Full-surface SMC & Reaching gain $K$ & 800 & \si{\degree\per\second\squared} & boundary-layer reaching law \\
  Full-surface SMC & Boundary layer $\phi$ & 50 & \si{\degree\per\second} & reduces chattering \\
  Full-surface SMC & Coupling hard limit $k_{\max}$ & 1.20 & -- & prevents ill-conditioning \\
  Full-surface SMC & Denominator margin $\mathrm{den}_{\min}$ & 0.60 & -- & enforces $B\cos\alpha-k\ge\mathrm{den}_{\min}$ \\
  Full-surface SMC & Coupling ramp duration & 200 & ms & ramp-in after engage \\
  Full-surface SMC & Internal $\dot{\alpha}$ clamp & 200 & \si{\degree\per\second} & \texttt{SMC4\_ALPHA\_DOT\_CTRL\_CLAMP\_DEG\_S} \\
  Full-surface SMC & Error clamp $|\theta_{\mathrm{err}}|$ & 78 & \si{\degree} & \texttt{SMC\_FULL\_THETA\_ERR\_CLAMP\_DEG} \\
  Full-surface SMC & Error clamp $|\dot{\theta}_{\mathrm{err}}|$ & 200 & \si{\degree\per\second} & \texttt{SMC\_FULL\_THETADOT\_ERR\_CLAMP\_DEG\_S} \\
  Full-surface SMC & Base term clamp & 300 & \si{\degree\per\second} & \texttt{SMC\_FULL\_THETA\_TERM\_CLAMP\_DEG\_S} \\
  Full-surface SMC & Extra accel cap & 12000 & \si{\step\per\second\squared} & \texttt{SMC4\_MAX\_ACC\_STEPS} \\
  Full-surface SMC & Extra speed cap & 2000 & \si{\step\per\second} & \texttt{SMC4\_MAX\_SPEED\_HZ} \\
  Full-surface SMC & Base assist gate $|\dot{\alpha}|$ & 250 & \si{\degree\per\second} & \texttt{SMC4\_BASE\_GATE\_ALPHADOT\_WINDOW\_DEG\_S} \\

  \addlinespace
  Reference tracking & Target limit $|\theta_{\mathrm{target}}|$ & 35 & \si{\degree} & \texttt{THETA\_TARGET\_LIMIT\_DEG} \\
  Reference tracking & Max reference speed & 12 & \si{\degree\per\second} & trapezoidal profile limit \\
  Reference tracking & Max reference acceleration & 60 & \si{\degree\per\second\squared} & trapezoidal profile limit \\
  Reference tracking & Stability gate $|\alpha|$ & 5 & \si{\degree} & \texttt{MOVE\_ALPHA\_WINDOW\_DEG} \\
  Reference tracking & Stability gate $|\dot{\alpha}|$ & 80 & \si{\degree\per\second} & \texttt{MOVE\_ALPHADOT\_WINDOW\_DEG\_S} \\
  Reference tracking & Engage grace & 200 & ms & \texttt{MOVE\_ENGAGE\_GRACE\_MS} before gating \\
  Reference tracking & Pause/resume debounce & 10 & ticks & \texttt{MOVE\_STABLE\_TICKS\_REQ} and \texttt{MOVE\_UNSTABLE\_TICKS\_REQ} \\
\end{longtable}
\endgroup

\section{Logging artifacts}
Each logger session produces:
\begin{itemize}
  \item a device-time time-series log (approximately \SI{50}{Hz}),
  \item a host-time event log (status + diagnostic messages).
\end{itemize}
Within a session, each occurrence of controller engagement followed by a disarm event defines one \emph{trial} (one ``engaged block'').

\section{Event-to-CSV alignment and limitations}
The event log timestamps are host wall-clock time, while the time-series log uses device time. Therefore, naive alignment (e.g., anchoring to an engage printout) can introduce \SI{100}{ms}--\SI{500}{ms} shifts and corrupt any time-window metrics.

The analysis pipeline aligns host events to device time by matching sparse status snapshots to the nearest device-time samples using a multi-signal error metric (e.g., $\alpha$, $\dot{\alpha}$, $\theta$, $\dot{\theta}$, acceleration, velocity) under a monotonicity constraint. A linear host-to-device mapping is then fit and residuals are reported. If alignment quality is unreliable, the pipeline avoids drawing precise event-time annotations and instead reports metrics computed directly from device-time trajectories.

\section{Metric definitions}
All metrics are defined once here and reused in tables:
\begin{itemize}
  \item \textbf{RMS and max}: $\mathrm{RMS}(|\alpha|)$, $\max|\alpha|$, $\max|\theta|$, etc., computed over the trial duration.
  \item \textbf{Drift slope}: slope of $\theta(t)$ over the trial (least-squares fit), reported in \si{\degree\per\second}.
  \item \textbf{Effort}: RMS and max of the commanded base acceleration $u$ (\si{\step\per\second\squared}).
  \item \textbf{Saturation/clamp percentages}: fraction of samples where acceleration saturation is active and fraction where the base is near the mechanical limit (clamped flag).
\end{itemize}

\paragraph{Spectral plots.}
Because CSV logging is approximately \SI{50}{Hz}, any spectrum is limited to low-frequency content ($\leq\SI{25}{Hz}$). These plots are included only as low-frequency indicators and are \emph{not} used to claim high-frequency SMC chattering characteristics.

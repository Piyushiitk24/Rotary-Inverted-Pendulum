\chapter{Problem Framing, Contributions, and Reading Guide}
\label{ch:framing}

\section{Problem}
This project studies upright-only stabilization of a rotary (Furuta) inverted pendulum under real hardware constraints (stepper actuation, sensor noise, limits, and real-time timing).

\section{Key contributions (summary)}
\begin{itemize}
  \item \textbf{Acceleration-input control matched to stepper actuation.} Most Furuta pendulum treatments model motor torque as the input; here we derive and implement an acceleration-input formulation $u=\ddot{\theta}$ to match stepper velocity-mode actuation (acceleration integrated to a commanded step rate).
  \item \textbf{First-principles modelling} of the custom rig (geometry, inertia, identified constants), with the full derivation included in the main body (Chapter~\ref{ch:modelling}).
  \item \textbf{Three upright controllers} implemented and experimentally compared:
    \begin{itemize}
      \item Linear full-state feedback (baseline)
      \item Hybrid nonlinear sliding mode controller (pendulum surface + gated base-centering assist)
      \item Full-surface sliding mode controller (single surface with $\theta$-terms)
    \end{itemize}
  \item \textbf{Reference tracking while balancing (“nudge mode”).} A trapezoidal $\theta_{\text{ref}}$ generator with pause/resume enables servo-like base positioning while maintaining upright balance.
  \item \textbf{Firmware robustness features} for a stepper-controlled rig: derivative filtering, wrap-safe angle differences, glitch rejection, engage gating, limits/soft limits, stall detection, and controlled disarm behavior.
\end{itemize}

\section{Reading guide}
\begin{itemize}
  \item Hardware and constraints: Chapter~\ref{ch:hardware}
  \item Notation and units: Chapter~\ref{ch:notation}
  \item Modelling derivation (main body): Chapter~\ref{ch:modelling}
  \item Firmware pipeline: Chapters~\ref{ch:measurement}--\ref{ch:actuation}
  \item Controllers: Chapters~\ref{ch:lin}--\ref{ch:smc4}
  \item Experiment protocol + metric definitions: Chapter~\ref{ch:experiments}
  \item Results + comparison: Chapter~\ref{ch:results}
\end{itemize}

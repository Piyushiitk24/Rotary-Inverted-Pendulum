\chapter{Results and Discussion}
\label{ch:results}

This chapter presents quantitative comparisons supported by representative time-series figures and summary tables generated from the recorded experimental dataset using the analysis pipeline described in Chapter~\ref{ch:experiments}.

\section{How to read this chapter}
The dataset used in this report consists of seven pinned representative trials: H1--H3 (hold), N1 (reference tracking / nudge), and T1--T3 (tap). Each condition is represented by one trial; comparisons in this chapter are therefore qualitative and metric-based rather than statistical.

All metrics referenced in this chapter are defined in Chapter~\ref{ch:experiments}. Each pinned trial is illustrated using a single time-domain plot showing $\alpha(t)$, $\theta(t)$, and the commanded base acceleration $u(t)$. For reference tracking (N1), the commanded $\theta_{\mathrm{ref}}(t)$ is also overlaid.

\section{Trials overview}
\noindent Table~\ref{tab:trials_overview} lists the seven pinned trials used in this report.
\InputIfExists{tables/tab_trials_overview.tex}{\emph{(Run export script to generate tables.)}}
\clearpage

\section{Upright hold-at-center (H1--H3)}
\noindent\textbf{Objective.} Each controller is engaged with $\theta_{\mathrm{ref}} \equiv 0$ (arm centered) and asked to balance upright without drifting toward the mechanical base limits.

\noindent\textbf{Key takeaways.} In the pinned hold trials (Table~\ref{tab:summary_hold}):
\begin{itemize}
  \item \textbf{Linear} provides the calmest steady balancing (lowest RMS upright error).
  \item \textbf{Hybrid SMC} provides the best peak suppression (smallest max $|\alpha|$ in the representative trial).
  \item \textbf{Full-surface SMC} stabilizes upright but demands substantially higher effort and exhibits measurable centering drift on this stepper-driven platform.
\end{itemize}
\noindent\textbf{Summary metrics.}
\InputIfExists{tables/tab_summary_hold.tex}{\emph{(Run export script to generate tables.)}}

\noindent\textbf{Representative time series.} Figures~\ref{fig:hold_lin_timeseries}--\ref{fig:hold_smc4_timeseries} illustrate steady upright balancing and the corresponding actuation demand.
\ThesisFigure{figures/hold_LIN_timeseries}{Linear hold-at-center representative trial.}{fig:hold_lin_timeseries}{0.95\linewidth}
\ThesisFigure{figures/hold_SMC_timeseries}{Hybrid SMC hold-at-center representative trial.}{fig:hold_smc_timeseries}{0.95\linewidth}
\ThesisFigure{figures/hold_SMC4_timeseries}{Full-surface SMC hold-at-center representative trial.}{fig:hold_smc4_timeseries}{0.95\linewidth}

\noindent\textbf{Interpretation.} The qualitative tradeoff (smoothness vs.\ peak suppression vs.\ effort) is consistent with the differing controller structures and with comparisons between linear and sliding-mode approaches in inverted pendulum studies \cite{Irfan2018,Hamza2019}.
\clearpage

\section{Commanded reference tracking / nudge mode (N1)}
\noindent\textbf{Objective.} Demonstrate balance-with-servo behavior by commanding a sequence of base targets $\theta_{\mathrm{ref}}(t)$ while maintaining upright balance. Only the linear controller is evaluated for reference tracking in the pinned dataset.

\noindent\textbf{Key takeaways.} The pinned run tracks the commanded sequence with finite rise times and limited overshoot while keeping induced pendulum motion moderate (Table~\ref{tab:nudge_steps}).

\noindent\textbf{Step metrics.}
\InputIfExists{tables/tab_nudge_steps.tex}{\emph{(Run export script to generate tables.)}}
\noindent\emph{Note:} Settle time uses a fixed $\pm\SI{1}{\degree}$ band; a value of ``--'' indicates that the criterion was not met within the available post-step window.

\noindent\textbf{Representative time series.} Figure~\ref{fig:nudge_lin_tracking} overlays the commanded target sequence and measured $\theta(t)$ and shows the corresponding $\alpha(t)$ response and actuation effort during motion.
\ThesisFigure{figures/nudge_LIN_nudge}{Linear commanded base reference tracking while balancing (nudge mode), including actuation effort $u(t)$.}{fig:nudge_lin_tracking}{0.95\linewidth}
\clearpage

\section{Finger tap disturbance (T1--T3)}
\noindent\textbf{Objective.} While balancing upright, the pendulum is disturbed by brief manual taps. The goal is to recover without falling or hitting base limits, probing robustness to unmodeled impulses and actuator limits.

\noindent\textbf{Key takeaways.} In the pinned tap trials (Table~\ref{tab:summary_tap}):
\begin{itemize}
  \item \textbf{Linear} remains upright for the full trial duration.
  \item \textbf{Hybrid SMC} and \textbf{Full-surface SMC} eventually fall in these representative runs under aggressive disturbance recovery.
  \item Both sliding-mode variants command larger acceleration effort during recovery, indicating more time spent near actuator limits.
\end{itemize}

\noindent\textbf{Summary metrics.}
\InputIfExists{tables/tab_summary_tap.tex}{\emph{(Run export script to generate tables.)}}

\noindent\textbf{Representative time series.} Figures~\ref{fig:tap_lin_timeseries}--\ref{fig:tap_smc4_timeseries} show disturbance excursions and recovery attempts.
\ThesisFigure{figures/tap_LIN_timeseries}{Linear disturbance representative trial.}{fig:tap_lin_timeseries}{0.95\linewidth}
\ThesisFigure{figures/tap_SMC_timeseries}{Hybrid SMC disturbance representative trial.}{fig:tap_smc_timeseries}{0.95\linewidth}
\ThesisFigure{figures/tap_SMC4_timeseries}{Full-surface SMC disturbance representative trial.}{fig:tap_smc4_timeseries}{0.95\linewidth}

\noindent\textbf{Interpretation.} The sliding-mode controllers compute large acceleration commands during recovery to drive the surface back toward zero. On this stepper-driven platform the actuator is open-loop: if the motor stalls or misses steps, the base does not realize the commanded motion, and the controller will continue to demand large effort based on the measured state. This feedback-through-actuation-limit mechanism can lock recovery into a high-effort regime until a base limit is approached or the pendulum leaves the upright window. Therefore the observed tap behavior reflects practical actuator constraints and saturation dynamics rather than a contradiction of sliding-mode robustness theory.
\clearpage

\section{Overall discussion and recommendations}
Based on this rig and the pinned dataset:
\begin{itemize}
  \item For calm steady hold-at-center behavior with low effort, the linear controller is the most consistent choice.
  \item If peak suppression is the priority in benign conditions, hybrid SMC can reduce peak excursions but can demand more aggressive effort during recovery.
  \item Full-surface SMC provides a unified nonlinear structure but is the most sensitive to actuator limits on a stepper-driven platform in these experiments.
\end{itemize}
Because each condition is represented by one trial, these conclusions are qualitative; statistical claims would require repeated trials per condition.

\chapter{Results and Discussion}
\label{ch:results}

This chapter is plot-driven. Figures and tables are generated from the recorded experimental dataset using a reproducible analysis pipeline (Chapter~\ref{ch:experiments}).

\section{Upright hold-at-center (H1--H3)}
\noindent\textbf{What was tested.} Each controller was engaged with $\theta_{\mathrm{ref}} \equiv 0$ (arm centered) and the pendulum held upright. The goal is to maintain upright balance without drifting toward the mechanical base limits.

\noindent\textbf{What to look for.} In the time series plots, compare (i) the magnitude of $\alpha(t)$ and $\dot{\alpha}(t)$ (upright quality), (ii) the drift of $\theta(t)$ (centering), and (iii) the command magnitude $u(t)$ (effort). Table~\ref{tab:summary_hold} summarizes the key metrics.
\ThesisFigure{figures/hold_LIN_timeseries}{Linear hold-at-center representative trial.}{fig:hold_lin_timeseries}{0.95\linewidth}
\ThesisFigure{figures/hold_SMC_timeseries}{Hybrid SMC hold-at-center representative trial.}{fig:hold_smc_timeseries}{0.95\linewidth}
\ThesisFigure{figures/hold_SMC4_timeseries}{Full-surface SMC hold-at-center representative trial.}{fig:hold_smc4_timeseries}{0.95\linewidth}

\noindent\textbf{Key takeaways (hold).} All three controllers stabilize upright in undisturbed conditions. In this dataset, the linear controller achieves the lowest RMS $\alpha$ (best ``quiet'' upright), the hybrid SMC achieves the lowest peak $|\alpha|$ (best peak suppression), and the full-surface controller requires substantially higher RMS acceleration (more aggressive effort) and exhibits measurable base drift. These differences are consistent with the structure of the controllers: SMC variants explicitly invert the nonlinear drift terms, while the full-surface surface couples centering and recovery and can demand larger accelerations when $\theta_{\mathrm{err}}$ is nonzero.

\subsection{Phase portraits ($\alpha$ vs $\dot{\alpha}$)}
\noindent The phase portraits summarize the upright behavior in the $(\alpha,\dot{\alpha})$ plane. A tighter cluster around $(0,0)$ indicates smaller oscillations and calmer recovery after disturbances within the trial window.
\begin{figure}[htbp]
  \centering
  \begin{subfigure}{0.32\linewidth}
    \centering
    \IncludeGraphicAuto{figures/hold_LIN_phase_alpha}{\linewidth}
    \caption{Linear (H1)}
  \end{subfigure}
  \begin{subfigure}{0.32\linewidth}
    \centering
    \IncludeGraphicAuto{figures/hold_SMC_phase_alpha}{\linewidth}
    \caption{Hybrid SMC (H2)}
  \end{subfigure}
  \begin{subfigure}{0.32\linewidth}
    \centering
    \IncludeGraphicAuto{figures/hold_SMC4_phase_alpha}{\linewidth}
    \caption{Full-surface SMC (H3)}
  \end{subfigure}
  \caption{Hold-at-center phase portraits.}
  \label{fig:hold_phase_alpha}
\end{figure}

\subsection{Control effort distributions}
\noindent The effort histograms show the distribution of commanded base acceleration $u(t)$ (steps/s$^2$). A narrower distribution and smaller tails indicate smoother actuation demands on the stepper pipeline.
\begin{figure}[htbp]
  \centering
  \begin{subfigure}{0.32\linewidth}
    \centering
    \IncludeGraphicAuto{figures/hold_LIN_effort_hist}{\linewidth}
    \caption{Linear (H1)}
  \end{subfigure}
  \begin{subfigure}{0.32\linewidth}
    \centering
    \IncludeGraphicAuto{figures/hold_SMC_effort_hist}{\linewidth}
    \caption{Hybrid SMC (H2)}
  \end{subfigure}
  \begin{subfigure}{0.32\linewidth}
    \centering
    \IncludeGraphicAuto{figures/hold_SMC4_effort_hist}{\linewidth}
    \caption{Full-surface SMC (H3)}
  \end{subfigure}
  \caption{Hold-at-center control effort histograms (commanded base acceleration in steps/s$^2$).}
  \label{fig:hold_effort_hist}
\end{figure}

\subsection{Low-frequency spectrum of $\alpha$}
\noindent These spectra are limited to $\leq\SI{25}{Hz}$ by the logging Nyquist rate. They are included only to compare low-frequency oscillations, not to claim high-frequency SMC chattering characteristics.
\begin{figure}[htbp]
  \centering
  \begin{subfigure}{0.32\linewidth}
    \centering
    \IncludeGraphicAuto{figures/hold_LIN_alpha_spectrum}{\linewidth}
    \caption{Linear (H1)}
  \end{subfigure}
  \begin{subfigure}{0.32\linewidth}
    \centering
    \IncludeGraphicAuto{figures/hold_SMC_alpha_spectrum}{\linewidth}
    \caption{Hybrid SMC (H2)}
  \end{subfigure}
  \begin{subfigure}{0.32\linewidth}
    \centering
    \IncludeGraphicAuto{figures/hold_SMC4_alpha_spectrum}{\linewidth}
    \caption{Full-surface SMC (H3)}
  \end{subfigure}
  \caption{Hold-at-center low-frequency spectra of $\alpha(t)$ (limited to $\leq\SI{25}{Hz}$ by logging Nyquist).}
  \label{fig:hold_alpha_spectrum}
\end{figure}

\subsection{Full-surface SMC diagnostics (sparse, 1 Hz)}
\noindent For the full-surface controller, additional sparse diagnostics are logged at \SI{1}{Hz}. In particular, the denominator margin and coupling ramp are important in explaining stability and failures: if the effective denominator shrinks, the same surface error produces larger acceleration commands.
\begin{figure}[htbp]
  \centering
  \begin{subfigure}{0.48\linewidth}
    \centering
    \IncludeGraphicAuto{figures/hold_SMC4_sparse_s}{\linewidth}
    \caption{$s(t)$ (sparse)}
  \end{subfigure}
  \begin{subfigure}{0.48\linewidth}
    \centering
    \IncludeGraphicAuto{figures/hold_SMC4_sparse_den}{\linewidth}
    \caption{$\mathrm{den}(t)$ (sparse)}
  \end{subfigure}

  \begin{subfigure}{0.48\linewidth}
    \centering
    \IncludeGraphicAuto{figures/hold_SMC4_sparse_kEff}{\linewidth}
    \caption{$k_{\mathrm{eff}}(t)$ (sparse)}
  \end{subfigure}
  \begin{subfigure}{0.48\linewidth}
    \centering
    \IncludeGraphicAuto{figures/hold_SMC4_sparse_kRamp}{\linewidth}
    \caption{$k_{\mathrm{ramp}}(t)$ (sparse)}
  \end{subfigure}
  \caption{Full-surface SMC sparse diagnostics captured from periodic status snapshots logged at \SI{1}{Hz}.}
  \label{fig:hold_smc4_sparse}
\end{figure}

\subsection{Summary table (hold)}
\InputIfExists{tables/tab_summary_hold.tex}{\emph{(Run export script to generate tables.)}}

\section{Commanded reference tracking / nudge mode (N1)}
\noindent\textbf{What was tested.} The linear controller is used as a balance-with-servo controller: while maintaining upright balance, the base reference $\theta_{\mathrm{ref}}(t)$ is commanded through a sequence of angle targets via an acceleration/velocity-limited trapezoidal profile.

\noindent\textbf{What to look for.} The tracking plot overlays $\theta(t)$ and the commanded target sequence. Compare rise time, overshoot, and steady-state error, and observe the induced pendulum deviation $\alpha(t)$ during motion. Table~\ref{tab:nudge_steps} summarizes step metrics for the commanded targets.
\ThesisFigure{figures/nudge_LIN_nudge}{Linear commanded base reference tracking while balancing (nudge mode).}{fig:nudge_lin_tracking}{0.95\linewidth}
\ThesisFigure{figures/nudge_LIN_timeseries}{Linear nudge run time series.}{fig:nudge_lin_timeseries}{0.95\linewidth}

\subsection{Phase portrait ($\alpha$ vs $\dot{\alpha}$)}
\noindent The nudge phase portrait illustrates how much pendulum motion is induced by commanded base motion. A well-behaved servo-like response keeps the trajectory close to upright even while the base transitions between targets.
\ThesisFigure{figures/nudge_LIN_phase_alpha}{Nudge run phase portrait.}{fig:nudge_phase_alpha}{0.75\linewidth}

\subsection{Control effort distribution}
\noindent This histogram shows how the commanded base acceleration distribution changes during combined balancing + reference tracking. Compared to the pure hold case, nudge mode naturally increases low-frequency effort to execute the trapezoidal motion profile.
\ThesisFigure{figures/nudge_LIN_effort_hist}{Nudge run control effort histogram (commanded base acceleration).}{fig:nudge_effort_hist}{0.75\linewidth}

\subsection{Low-frequency spectrum of $\alpha$}
\noindent The low-frequency spectrum highlights whether commanded base motion introduces additional oscillatory content in $\alpha(t)$ within the measurable bandwidth.
\ThesisFigure{figures/nudge_LIN_alpha_spectrum}{Nudge run low-frequency spectrum of $\alpha(t)$ (limited to $\leq\SI{25}{Hz}$ by logging Nyquist).}{fig:nudge_alpha_spectrum}{0.75\linewidth}

\subsection{Step metrics table (nudge)}
\InputIfExists{tables/tab_nudge_steps.tex}{\emph{(Run export script to generate tables.)}}
\noindent\emph{Note:} ``Settle time'' uses a fixed $\pm\SI{1}{\degree}$ criterion. A value of ``--'' indicates that the criterion was not met within the available post-step window.

\section{Finger tap disturbance (T1--T3)}
\noindent\textbf{What was tested.} While balancing upright, the pendulum is disturbed by brief finger taps. The goal is to recover without falling or hitting base limits. This experiment probes robustness to unmodeled impulses and actuator limits.

\noindent\textbf{What to look for.} Compare peak $|\alpha|$ during recovery, how quickly $|\alpha|$ returns near zero, and whether the base approaches the mechanical limits. Table~\ref{tab:summary_tap} summarizes the outcomes in the pinned dataset.
\ThesisFigure{figures/tap_LIN_timeseries}{Linear disturbance representative trial.}{fig:tap_lin_timeseries}{0.95\linewidth}
\ThesisFigure{figures/tap_SMC_timeseries}{Hybrid SMC disturbance representative trial.}{fig:tap_smc_timeseries}{0.95\linewidth}
\ThesisFigure{figures/tap_SMC4_timeseries}{Full-surface SMC disturbance representative trial.}{fig:tap_smc4_timeseries}{0.95\linewidth}

\noindent\textbf{Key takeaways (tap).} In the pinned tap dataset, the linear controller remains upright for the full trial duration, while both sliding-mode variants eventually fall. This does not contradict the theoretical robustness of SMC; instead it highlights a practical limitation of stepper actuation under aggressive disturbance recovery: if the commanded acceleration/speed cannot be realized (missed steps or stall ``ticks''), the effective plant deviates from the assumed acceleration-input channel and recovery can fail even when $\alpha$ remains within the nominal upright window. The full-surface controller additionally couples centering into the surface, which can increase required acceleration during taps.

\subsection{Phase portraits ($\alpha$ vs $\dot{\alpha}$)}
\noindent The phase portraits visualize the disturbance excursion and recovery trajectory in the pendulum subspace. Wider loops correspond to larger impulse responses and/or slower return to the upright neighborhood.
\begin{figure}[htbp]
  \centering
  \begin{subfigure}{0.32\linewidth}
    \centering
    \IncludeGraphicAuto{figures/tap_LIN_phase_alpha}{\linewidth}
    \caption{Linear (T1)}
  \end{subfigure}
  \begin{subfigure}{0.32\linewidth}
    \centering
    \IncludeGraphicAuto{figures/tap_SMC_phase_alpha}{\linewidth}
    \caption{Hybrid SMC (T2)}
  \end{subfigure}
  \begin{subfigure}{0.32\linewidth}
    \centering
    \IncludeGraphicAuto{figures/tap_SMC4_phase_alpha}{\linewidth}
    \caption{Full-surface SMC (T3)}
  \end{subfigure}
  \caption{Tap-disturbance phase portraits.}
  \label{fig:tap_phase_alpha}
\end{figure}

\subsection{Control effort distributions}
\noindent The effort histograms provide a compact view of how hard each controller drives the stepper during disturbance recovery. Heavy tails near saturation are an indicator of actuator stress and a higher probability of missed steps.
\begin{figure}[htbp]
  \centering
  \begin{subfigure}{0.32\linewidth}
    \centering
    \IncludeGraphicAuto{figures/tap_LIN_effort_hist}{\linewidth}
    \caption{Linear (T1)}
  \end{subfigure}
  \begin{subfigure}{0.32\linewidth}
    \centering
    \IncludeGraphicAuto{figures/tap_SMC_effort_hist}{\linewidth}
    \caption{Hybrid SMC (T2)}
  \end{subfigure}
  \begin{subfigure}{0.32\linewidth}
    \centering
    \IncludeGraphicAuto{figures/tap_SMC4_effort_hist}{\linewidth}
    \caption{Full-surface SMC (T3)}
  \end{subfigure}
  \caption{Tap-disturbance control effort histograms (steps/s$^2$).}
  \label{fig:tap_effort_hist}
\end{figure}

\subsection{Low-frequency spectrum of $\alpha$}
\noindent These spectra compare low-frequency oscillations following disturbance events. As in the hold case, they should not be interpreted as chattering bandwidth measurements.
\begin{figure}[htbp]
  \centering
  \begin{subfigure}{0.32\linewidth}
    \centering
    \IncludeGraphicAuto{figures/tap_LIN_alpha_spectrum}{\linewidth}
    \caption{Linear (T1)}
  \end{subfigure}
  \begin{subfigure}{0.32\linewidth}
    \centering
    \IncludeGraphicAuto{figures/tap_SMC_alpha_spectrum}{\linewidth}
    \caption{Hybrid SMC (T2)}
  \end{subfigure}
  \begin{subfigure}{0.32\linewidth}
    \centering
    \IncludeGraphicAuto{figures/tap_SMC4_alpha_spectrum}{\linewidth}
    \caption{Full-surface SMC (T3)}
  \end{subfigure}
  \caption{Tap-disturbance low-frequency spectra of $\alpha(t)$ (limited to $\leq\SI{25}{Hz}$).}
  \label{fig:tap_alpha_spectrum}
\end{figure}

\subsection{Full-surface SMC diagnostics (sparse, 1 Hz)}
\noindent The sparse diagnostics help interpret full-surface behavior under taps. In particular, the denominator margin and the effective coupling weight explain when the controller becomes ``stiffer'' (larger acceleration for similar surface errors).
\begin{figure}[htbp]
  \centering
  \begin{subfigure}{0.48\linewidth}
    \centering
    \IncludeGraphicAuto{figures/tap_SMC4_sparse_s}{\linewidth}
    \caption{$s(t)$ (sparse)}
  \end{subfigure}
  \begin{subfigure}{0.48\linewidth}
    \centering
    \IncludeGraphicAuto{figures/tap_SMC4_sparse_den}{\linewidth}
    \caption{$\mathrm{den}(t)$ (sparse)}
  \end{subfigure}

  \begin{subfigure}{0.48\linewidth}
    \centering
    \IncludeGraphicAuto{figures/tap_SMC4_sparse_kEff}{\linewidth}
    \caption{$k_{\mathrm{eff}}(t)$ (sparse)}
  \end{subfigure}
  \begin{subfigure}{0.48\linewidth}
    \centering
    \IncludeGraphicAuto{figures/tap_SMC4_sparse_kRamp}{\linewidth}
    \caption{$k_{\mathrm{ramp}}(t)$ (sparse)}
  \end{subfigure}
  \caption{Full-surface SMC sparse diagnostics during tap runs (status snapshots at \SI{1}{Hz}).}
  \label{fig:tap_smc4_sparse}
\end{figure}

\subsection{Summary table (tap)}
\InputIfExists{tables/tab_summary_tap.tex}{\emph{(Run export script to generate tables.)}}

\section{Trials overview}
\noindent Table~\ref{tab:trials_overview} lists the seven pinned trials used in this report. The dataset size is intentionally small (one representative run per controller per experiment type) and is sufficient for qualitative comparison and for demonstrating the implemented methods; statistical claims would require additional repeated trials per condition.
\InputIfExists{tables/tab_trials_overview.tex}{\emph{(Run export script to generate tables.)}}
